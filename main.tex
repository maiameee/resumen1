\documentclass[11pt,a4paper]{article}
\usepackage[utf8]{inputenc}
\usepackage{graphicx}
\usepackage[spanish]{babel} 
\usepackage{textcomp}
\usepackage{float}
\usepackage{subfig}
\usepackage{multicol}
\usepackage[left=2cm,top=2.5cm,right=2cm,bottom=2.5cm]{geometry} 
\usepackage{lipsum}
\usepackage{circuitikz}
\usepackage{fancyhdr}
\usepackage{amsmath}
\begin{document}

\title{\Huge Resumen de Anisotropic fluid dynamics fo gubser flow}
\author{\huge  arXiv:1703.10955 [nucl-th]}





\date{Abril 2017}

\renewcommand{\headrulewidth}{0.5pt}




\maketitle

\section{Hidrodinámica viscosa}

Para obtener la hidrodinámica viscosa estándar expanden la función distribución cinética alrededor del equilibrio local.

\begin{equation}
f( \hat{x}, \hat{p})= f_{eq}(\beta_{\hat{u}}(\hat{x})(-\hat{u}(\hat{x}).\hat{p}))+{\delta}f( \hat{x}, \hat{p})
\label{expansion}
\end{equation}

Con $\beta_{\hat{u}}=\frac{1}{\hat{T}(\hat{x})}$ y en el sistema de referencia local (LRF) $-\hat{u}(\hat{x}).\hat{p}$ es isótropo en el espacio de los momentos. Acá ${\delta}f$ es el que llevas la información las desviaciones del equilibrio local, en particular de las desviaciones del las anisotropías locales causadas por una expansión global anisótropa.



En el marco de Landau la forma mas general del tensor de energía-momento es:

\begin{equation}
\hat{T}^{{\mu}{\nu}}= \hat{\epsilon}\hat{u}^{\mu}\hat{u}^{\nu}+\hat{P}\hat{\Delta}^{{\mu}{\nu}}+\hat{\pi}^{{\mu}{\nu}}
\end{equation}

Con $\hat{\epsilon}$ la densidad de energía en LRF, $\hat{P}=\hat{P}_{0}+\hat{\Pi}$ es la presión isótropa y $\hat{\pi}^{{\mu}{\nu}}$ es el tensor de esfuerzos viscosos.

Para sistemas con simetría conforme la presión de bulk $\hat{\Pi}$ es nula. Esto se debe a que la presión de bulk es una fuerza viscosa que se opone a la expansión o compresión del fluido, lo cual se contradice con la simetría conforme (o al revés). En estos casos la presión isótropa  $\hat{P}$ es la presión térmica $\hat{P}_{0}=\hat{\epsilon}/3$.

Las cantidades macroscópicas que aparecen en el tensor de energía-momento son proyecciones de los momentos de la función distribución $f( \hat{x}, \hat{p})$

\begin{equation}
\hat{\epsilon}=\hat{u}_{\mu}\hat{u}_{\nu}\hat{T}^{{\mu}{\nu}}=<(\hat{u}(\hat{x}).\hat{p})^2>
\end{equation}
\begin{equation}
\hat{P}=\frac{1}{3}\hat{\Delta}_{{\mu}{\nu}}\hat{T}^{{\mu}{\nu}}= \frac{1}{3}\hat{\Delta}_{{\mu}{\nu}}\hat{p}^{\mu}\hat{p}^{\nu}
\end{equation}
\begin{equation}
\hat{\pi}^{{\mu}{\nu}}= \hat{T}^{<{\mu}{\nu}>}=<\hat{p}^{<\mu}\hat{p}^{\nu>}>
\end{equation}

Donde $-\hat{u}_{\mu}\hat{u}_{\nu}$ es un proyector en la dirección de la cuadrivelocidad y $\hat{\Delta}_{{\mu}{\nu}}=\hat{g}_{{\mu}{\nu}}+\hat{u}_{\mu}\hat{u}_{\nu}$  proyecta en la superficie ortogonal a $\hat{u}_{\mu}$. Debido a que en el LRF tenemos $\hat{u}_{\mu}=(1,0,0,0)$, estos proyectores se llaman locamente temporal y espacial respectivamente.
Por último el tensor de esfuerzos viscosos $\hat{\pi}^{{\mu}{\nu}}$ es la proyección de $\hat{T}^{{\mu}{\nu}}$ ortogonal a $\hat{u}_{\mu}$ y sin traza. Este proyector tal que $\hat{\Delta}^{{\mu}{\nu}}_{{\alpha}{\beta}}=\frac{1}{2}(\hat{\Delta}^{\mu}_{\alpha}\hat{\Delta}^{\nu}_{\beta}+\hat{\Delta}^{\mu}_{\beta}\hat{\Delta}^{\nu}_{\alpha}-\frac{2}{3}\hat{\Delta}^{{\mu}{\nu}}\hat{\Delta}_{{\alpha}{\beta}})$ y $\hat{B}^{<{\mu}{\nu}>}=\hat{\Delta}^{{\mu}{\nu}}_{{\alpha}{\beta}}\hat{B}^{{\mu}{\nu}}$.

\textbf{La unicidad de la descomposición (\ref{expansion}) requiere fijar la inversa de la temperatura local $\beta_{\hat{u}}(\hat{x})$}, lo cual se realiza gracias al la condición de Landau:

\begin{equation}
\hat{\epsilon}=<(\hat{u}(\hat{x}).\hat{p})^2>_{eq}=\hat{\epsilon}_{eq}({\hat{T}})=\frac{3}{\pi^2}{\hat{T}}^4
\end{equation}

De esta forma ${\hat{T}}$ en $f_{eq}$ es ajustada de forma que ${\delta}f$ no contribuya  a la densidad de energía en LRF $(\hat{\epsilon})$. Debido a esto toda la información de las desviaciones del sistema del equilibrio local esta guardada en el tensor de esfuerzos viscosos $\hat{\pi}^{{\mu}{\nu}}=<\hat{p}^{<\mu}\hat{p}^{\nu>}>_{{\delta}f}$ (es un momento de ${{\delta}f}$)


\subsection{Ecuación para la densidad de energía}
La ecuación de evolución para la densidad de energía se obtiene proyectando sobre $u_{nu}$ (proyección temporal) la ecuación de conservación del tensor energía-momento ($\hat{D}_{\mu}\hat{T}^{{\mu}{\nu}}=0$), donde $\hat{D}_{\mu}$ es la derivada covariante.
Usando la conexión de Levi-Civita, los símbolos de Christoffel la métrica de de Sitter no nulos son:

\begin{subequations}
\begin{align}
\Gamma^{0}_{{1}{1}}=\cosh(\rho)\sinh(\rho)\\
\Gamma^{0}_{{2}{2}}=\cosh(\rho)\sinh(\rho)\sin^2(\theta)\\
\Gamma^{1}_{{2}{2}}=\cos(\theta)\sin(\theta)\\
\Gamma^{1}_{{1}{0}}=\Gamma^{1}_{{0}{1}}=\Gamma^{1}_{{2}{0}}=\Gamma^{1}_{{0}{2}}=\tanh(\rho)\\
\Gamma^{2}_{{2}{1}}=\Gamma^{2}_{{1}{2}}=-\cot(\theta)
\end{align}
\end{subequations}

Utilizando que $u_{\nu}=(-1,0,0,0)$ la proyección queda:

\begin{equation}
u_{\nu}\hat{D}_{\mu}\hat{T}^{{\mu}{\nu}}= \hat{T}^{{\mu}{0}}_{,\mu} +\Gamma^{\mu}_{{j}{\mu}}\hat{T}^{{j}{0}}+\Gamma^{0}_{{j}{\mu}}\hat{T}^{{\mu}{j}}
\end{equation}

Dado que $\hat{T}^{{\mu}{0}}=u_{\nu}\hat{T}^{{\mu}{\nu}}=\hat{\epsilon}\hat{u}^{\mu}$ el primer termino queda $\hat{T}^{{\mu}{0}}_{,\mu}={\partial}_{\rho}\hat{\epsilon}$. El segundo termino solo aporta el término $j=0$, $\Gamma^{\mu}_{{0}{\mu}}\hat{T}^{{0}{0}}=\Gamma^{1}_{{0}{1}}\hat{\epsilon}+\Gamma^{2}_{{0}{2}}\hat{\epsilon}=2\tanh(\rho)\hat{\epsilon}$.
Finalmente desarrollamos el tercer termino:

\begin{equation}
\Gamma^{0}_{{j}{\mu}}\hat{T}^{{\mu}{j}}=\hat{\epsilon}\Gamma^{0}_{{j}{\mu}}\hat{u}^{j}\hat{u}^{\nu}+\hat{P}\Gamma^{0}_{{j}{\mu}}\hat{\Delta}^{{\mu}{j}}+\Gamma^{0}_{{j}{\mu}}\hat{\pi}^{{\mu}{j}}
\label{tercterm}
\end{equation}

Se puede ver que el primer termino de (\ref{tercterm}) se anula ya que $\Gamma^{0}_{{j}{\mu}}\hat{u}^{j}\hat{u}^{\nu}=\Gamma^{0}_{{0}{0}}=0$. Se puede ver que el segundo termino tiene una parte que se anula ($\hat{P}\hat{u}^{j}\hat{u}^{\nu}$) cuando desarrollamos $\hat{\Delta}^{{\mu}{j}}$. El segundo término termina dando $\hat{P}\Gamma^{0}_{{j}{\mu}}\hat{\Delta}^{{\mu}{j}}=\hat{P}(\Gamma^{0}_{{1}{1}}\hat{g}^{{1}{1}}+\Gamma^{0}_{{2}{2}}\hat{g}^{{2}{2}})=2\tanh(\rho)\hat{P}_{0}=\frac{2}{3}2\tanh(\rho)\hat{\epsilon}$, donde en el ultimo paso se uso que $\hat{P}=\hat{P}_{0}$ cumple la ecuación de estado conforme.\\


Para analizar el ultimo término conviene desarrollar $\hat{\Delta}^{{\mu}{\nu}}_{{\alpha}{\beta}}$ teniendo en cuenta que $\hat{u}^{\nu}$





\begin{thebibliography}{10}

\bibitem{Durrer} R. Durrer, A. Neronov "Cosmological Magnetic Fields: Their Generation, Evolution and Observation" 	arXiv:1303.7121 [astro-ph.CO]Appl. 
%para citar es \cite{}

\end{thebibliography}



\end{document}